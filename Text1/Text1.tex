\documentclass[letterpaper,11pt]{article}

\usepackage{ucs}
\usepackage[utf8x]{inputenc}
\usepackage{graphicx}
\usepackage{amsfonts}
\usepackage{dsfont}
\usepackage{amssymb}
\usepackage{amsmath}
\usepackage{amsthm}
\usepackage{enumerate}
\usepackage{stmaryrd}
\usepackage{fullpage}
\usepackage{ifthen}
\usepackage{subfigure}
\usepackage{epic}
\usepackage{authblk}
\usepackage{textcomp}
\usepackage[small]{caption}


\usepackage[hypertexnames=false,colorlinks=true,linkcolor=blue,citecolor=blue]{hyperref}
\usepackage[numbers,comma,square,sort&compress]{natbib}
\usepackage[letterpaper,text={7in,9in},centering]{geometry}


\usepackage{color}
\usepackage{titlesec}
\setlength{\parindent}{0.0in}
\setlength{\parskip}{1.0ex plus0.2ex minus0.2ex}
\renewcommand{\baselinestretch}{1.1}
\graphicspath{{eps/}{pdf/}}
%\setcaptionmargin{0.25in}
\def\captionfont{\itshape\small}
\def\captionlabelfont{\upshape\small}

\renewcommand{\labelenumi}{(\roman{enumi})}

\newcommand{\bqq}{\begin{equation}}
\newcommand{\eqq}{\end{equation}}
\newcommand{\bqs}{\begin{equation*}}
\newcommand{\eqs}{\end{equation*}}

\newcommand{\C}{\mathbb{C}}
\newcommand{\D}{\mathbb{D}}
\newcommand{\N}{\mathbb{N}}
\newcommand{\R}{\mathbb{R}}
\newcommand{\Z}{\mathbb{Z}}
\newcommand{\T}{\mathbb{T}}

\newcommand{\rme}{\mathrm{e}}
\newcommand{\rmi}{\mathrm{i}}
\newcommand{\rmd}{\mathrm{d}}
\newcommand{\rmo}{{\scriptstyle\mathcal{O}}}
\newcommand{\rmO}{\mathcal{O}}
\newcommand{\eps}{\varepsilon}

\numberwithin{equation}{section}

% \newenvironment{Hypothesis}[1]%
%   {\begin{trivlist}\item[]{\bf Hypothesis #1 }\em}{\end{trivlist}}

\renewcommand{\arraystretch}{1.25}

% Define Theorem Styles ----------------------------------
\theoremstyle{plain}
\newtheorem{theorem}{Theorem}[section]
\newtheorem{proposition}[theorem]{Proposition}
\newtheorem{lemma}[theorem]{Lemma}
\newtheorem{corollary}[theorem]{Corollary}
\newtheorem{conjecture}[theorem]{Conjecture}
\newtheorem{main}[theorem]{Main Result}
\newtheorem{rmk}[theorem]{rmk}
\newtheorem{Hypothesis}[theorem]{Hypothesis}

\newcommand{\etal}{\textit{et al.}\ }

\begin{document}

\section{Relative equilibria}

Consider function $f(u, \theta, \delta, \varepsilon)$,
where $u, \delta, \varepsilon \in \R$ and $\theta \in \R^m/\Z^m$. For instance
$$
f(u, \theta, \delta, \varepsilon) = (u+ \delta)(1-u)(1+u) + \varepsilon g(\theta),
$$
where
$$
g(\theta) = \sum_{j = 1}^{m}g_j\cos(2 \pi \theta_j), \quad g_j \in \R.
$$

We study ``travelling waves'' of the equation
\begin{equation}\label{eq2.1}
  u_t = u_{xx} + f(u, \theta_0 + \omega x, \delta, \varepsilon).
\end{equation}

Let us rewrite this equation in the functional space
\begin{equation}\label{e:U}
\begin{cases}
\frac{d}{dt}U = A U + F(U, \theta, \delta, \varepsilon),\\
\frac{d}{dt}\theta = 0,
\end{cases}
\end{equation}
where $U\in BC^0_\mathrm{unif}(\R)$, $\theta\in\T^m$, $A=\partial_{xx}$, and
$$
F(U,\theta;\delta,\varepsilon)(x)=f(u(x),\theta + \omega x;\delta,\varepsilon).
$$
Define the action $\mathcal{T}_y$ of the additive group $\R$ on the phase space $BC^0\times \T^m$ via
\[
\mathcal{T}_y=\mathrm{diag}\,(T_y,S_y), \quad T_y u(\cdot)=u(\cdot -y),\quad  S_y \phi = \phi+\omega y.
\]
The system \eqref{e:U} is invariant under the diagonal action $(T,S)$ on $BC^0\times \T^m$.

For $\varepsilon = 0$ the equation \eqref{eq2.1} admits a travelling wave solution, which is a relative equilibria of \eqref{eq2.1}.

However, in the rigorous proof we are going to study equations \eqref{e:U} with action $(T_y, S_y)$. Action $S_y$ does not have fixed points. So any travelling wave is not a relative equilibria with respect to the symmetry $(T_y, S_y)$. 

We can generalise the notion of relative equilibria to the following. Let $U(t)$ be a travelling wave for $\varepsilon = 0$. Consider sets $A(t) = \{(U(t), \theta): \; \theta \in \T^m\}$. Then for any $t$ there exists $y$ such that $A(t) = \mathcal{T}_y A(0)$. So in some sense $A(0)$ is a relative equilibria. But to this notion we cannot apply [SSW97] and have to repeat its proof.

\end{document}